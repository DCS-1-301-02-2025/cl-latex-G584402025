\documentclass[a4paper,11pt,dvipdfmx]{ujarticle}
% パッケージ
\usepackage{graphicx}
\usepackage{url}
% レイアウト指定を記述したファイルの読み込み
\input{layout}

% タイトルと氏名を変更せよ.
\title{日本におけるデジタル化の状況}
\author{G584402025 小森尚斗}

\begin{document}

\maketitle %ここにタイトルが入る
\section{デジタル競争ランキング}
% ここから本文
国際経営開発研究所(LMD)の調査\cite{imd}によると、日本の
デジタル競争力ランキングは図\ref{fig:保有率}に示すように、調査対象
の64カ国中、総合で28位、準備分野で27位となっている。

\begin{figure}[htbp]
    \centering
    \includegraphics[width=0.7\linewidth]{fig51.png}
    \caption{情報通信機器の世帯保有率}\label{fig:保有率}
\end{figure}

% 節見出し: \section{}
% を使う

% 本文(1)
%  参考文献の参照: \cite{}
%  図番号の参照: \ref{}
% を使う
% 文献データベースのキーワードは oecd と imd
% になっている.

% 図の挿入
% \includegraphics{}
% を
% \begin{figure}[htbp]
% \end{figure}
% で囲み
% \caption{}
% で図のタイトルを入れる.
% \label{}
% を使って図番号が参照できるようにする
% また,
% \centering
% で図が中央に来るようにする

% ーーー
% 節見出し(2)

% 本文(2)
\section{ブローバンドの整備状況}
OECDによるブローバンド回線の普及に関する調査\cite{oecd}
によると、表\ref{tbl:利用状況}に表すように、日本における
100人あたりの光ファイバー回線の加入者は29.0で、韓国、スウェーデン、
ノルウェーに続いて第4位となっている。

\begin{table}[htbp]
    \centering
    \caption{光ファイバー回線の加入者数(100人あたり)}
    \label{tbl:利用状況}

    \begin{tabular}{|l|r|r|}\hline
        順位 & 国名 & 加入者数 \\
        \hline
        1位 & 韓国 & 38.2 \\
        \hline
        2位 & スウェーデン & 31.2 \\
        \hline
        3位 & ノルウェー & 29.5 \\
        \hline
        4位 & 日本 & 29.0 \\
        \hline
        5位 & アイスランド & 28.8 \\
        \hline
        6位 & スペイン & 27.3 \\
        \hline 
        7位 & ポルトガル & 25.1 \\
        \hline
         8位 & ニュージーランド & 23.6 \\
        \hline
         9位 & リトアニア & 22.3 \\
        \hline
         10位 & フランス & 21.2 \\
        \hline
    \end{tabular}
\end{table}
% 表の挿入
% \begin{tabular}
% \end{tabular}    
% による表の記述を 
% \begin{table}[htbp]
% \end{table}
% で囲み
% \caption{}
% で表のタイトルを入れる.
% \label{}
% を使って表番号が参照できるようにする
% また,
% \centering
% で表が中央に来るようにする
\section{考察}
\begin{itemize}
    \item  アメリカは光ファイバー普及率が低いが、デジタル競争ランキングでは1位である。
このことから一部の恵まれた層が飛び抜けていることが考えられる。
    \item  日本は光ファイバーの普及率が高いが、デジタル競争ランキングの
順位は低い。このことから恵まれた環境の活用の仕方がわからない人
が多いことが考えられる。
\end{itemize}

% ーーー
% 見出し(3)
% 考察
%
% \begin{itemize}
% \end{itemize}
% を使って箇条書きで記述する

% ここに参考文献が入る
%
\bibliographystyle{junsrt}
\bibliography{exercise.bib}

\end{document}